\documentclass[conference,a4paper]{IEEEtran}
\IEEEoverridecommandlockouts
% The preceding line is only needed to identify funding in the first footnote. If that is unneeded, please comment it out.
%Template version as of 6/27/2024

\usepackage{cite}
\usepackage{amsmath,amssymb,amsfonts}
\usepackage{algorithmic}
\usepackage{graphicx}

\usepackage{textcomp}
\usepackage{tikz}
\usepackage{booktabs}
\usepackage{xcolor}
\usepackage{hyperref}
\usepackage{array}
\usepackage{tabularx}
\usepackage{amssymb} 


\def\BibTeX{{\rm B\kern-.05em{\sc i\kern-.025em b}\kern-.08em
    T\kern-.1667em\lower.7ex\hbox{E}\kern-.125emX}}
\begin{document}

% Custom figure command: pass filename and caption
\newcommand{\cfigure}[2]{%
  \begin{figure}[h]
    \centering
    \includegraphics[width=\linewidth]{figures/#1.png}%
    \caption{#2}%
    \label{fig:#1}%
  \end{figure}%

}
\title{AI for disaster detection and response using satellite imagery}

\author{Dwayne Mark Acosta (300665276) \\ Mohamed Amine Benaziza (300684553) \\ David Franz (300360491) \\ Ray Marange (300671115) \\ James Thompson (300680096)\\
\textit{Victoria University of Wellington}\\}
\date{\today}

\maketitle

\section{Introduction}

% Introduce this paper, It is a summary of ai disaster detection and problems using 5 primary papers (but hopefully more) ...
%% introduce the what remote sensing is
%% tell some statistics about how disaster are bad and increasing (the papers have lots of references)
%% more or less show that there is a great opportunity and need for better disaster response

When disasters strike a rapid and effective response is critical to mitigate their impact on affect communities . Understanding the disaster is essential for coordinating an effective response. Most methods of understanding a disaster rely on ground based data collection which can be slow, dangerous and limited in scope \cite{nhess-21-1431-2021}. Remote sensing is the process of collecting data from a distance, often using satellites or aircraft. A challenge of using remote sensing is that the scale of data is that manual processing is too expensive and slow. By using AI to analyze remote sensing data we can automate the process of extracting useful information from the data. This paper will review the current state of the art in using AI for disaster detection and response.

We will look at how important it is to have effective disaster response and the what can happen as AI is used in the disaster response process. Then we will discuss different applications of AI and remote sensing for disaster response. Specifically the applications of classifying damaged buildings \cite{kimDisasterAssessmentUsing2022,teohExploringGenerativeAI2024,lagapEnhancingPostDisasterDamage2025} and detecting wildfires \cite{elbohy2025fusion,jiaoForestFirePatterns2023}. Lastly we will discuss the challenges of using AI for disaster response and some attempts or methods to overcome these challenges. The challenges mainly fall into two categories data availability and compute constraints.


\section{Implication of effective or ineffective disaster response}

% Disaster disproportionately effect the poor and economically vulnerable
% Can amplify existing inequalities in communities that don't have survellance or data collection. Or help mitagte it due to easier data collection
% Talking about how it could include socioeconomic data into the disaster response information (paper example)
% (paper example) from all of the introductions and discussion sections.

\section{Application of remote sensing in disaster response}

% This is the results section where we can describe the different application and some basic stats to show how well it can or cannot work (like in chinese mountain range paper)
% How we can use it to detect and classify damaged buildings (~3 paper examples)
% How we can use it for fire detection (2 paper examples)
% How we can use it for more general disaster detection (1 paper example)

\section{Challenges of using remote sensing for disaster response}

\subsection{Data availability}

% Using super sampling to make low res data more useful (paper example)
% Using genAI to make up synthetic data (paper example)

\subsubsection{Compute constraints}

% How it can be hard to run large models on limited hardware and bandwidth
% Embedded processing due to bandwidth contraints from satellites (paper example)
% Compute constraints on the ground and speed requirements (paper example)

\section{Conclusion}

% Summarise the main points
% That we can use remote sensing for disaster response
% That it can be great but there are still challenges
% The current challenges that it face
% Touch on future directions (like end to end disaster response, smaller models, more data sources etc)

\bibliographystyle{IEEEtran}
\bibliography{references}
  
\end{document}
