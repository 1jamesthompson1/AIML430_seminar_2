\documentclass[conference,a4paper]{IEEEtran}
\IEEEoverridecommandlockouts
% The preceding line is only needed to identify funding in the first footnote. If that is unneeded, please comment it out.
%Template version as of 6/27/2024

\usepackage{cite}
\usepackage{amsmath,amssymb,amsfonts}
\usepackage{algorithmic}
\usepackage{graphicx}

\usepackage{textcomp}
\usepackage{tikz}
\usepackage{booktabs}
\usepackage{xcolor}
\usepackage{hyperref}
\usepackage{array}
\usepackage{tabularx}
\usepackage{amssymb} 


\def\BibTeX{{\rm B\kern-.05em{\sc i\kern-.025em b}\kern-.08em
    T\kern-.1667em\lower.7ex\hbox{E}\kern-.125emX}}
\begin{document}

% Custom figure command: pass filename and caption
\newcommand{\cfigure}[2]{%
  \begin{figure}[h]
    \centering
    \includegraphics[width=\linewidth]{figures/#1.png}%
    \caption{#2}%
    \label{fig:#1}%
  \end{figure}%

}
\title{AI for disaster detection and response using satellite imagery}

\author{Dwayne Mark Acosta (300665276) \\ Mohamed Amine Benaziza (300684553) \\ David Franz (300360491) \\ Ray Marange (300671115) \\ James Thompson (300680096)\\
\textit{Victoria University of Wellington}\\}
\date{\today}

\maketitle

\section{Introduction}

% Introduce this paper, It is a summary of ai disaster detection and problems using 5 primary papers (but hopefully more) ...
%% introduce the what remote sensing is
%% tell some statistics about how disaster are bad and increasing (the papers have lots of references)
%% more or less show that there is a great opportunity and need for better disaster response

When disasters strike a rapid and effective response is critical to mitigate their impact on affect communities . Understanding the disaster is essential for coordinating an effective response. Most methods of understanding a disaster rely on ground based data collection which can be slow, dangerous and limited in scope \cite{nhess-21-1431-2021}. Remote sensing is the process of collecting data from a distance, often using satellites or aircraft. A challenge of using remote sensing is that the scale of data is that manual processing is too expensive and slow. By using AI to analyze remote sensing data we can automate the process of extracting useful information from the data. This paper will review the current state of the art in using AI for disaster detection and response.

We will look at how important it is to have effective disaster response and the what can happen as AI is used in the disaster response process. Then we will discuss different applications of AI and remote sensing for disaster response. Specifically the applications of classifying damaged buildings \cite{kimDisasterAssessmentUsing2022,teohExploringGenerativeAI2024,lagapEnhancingPostDisasterDamage2025} and detecting wildfires \cite{elbohy2025fusion,jiaoForestFirePatterns2023}. Lastly we will discuss the challenges of using AI for disaster response and some attempts or methods to overcome these challenges. The challenges mainly fall into two categories data availability and compute constraints.


\section{Implication of effective or ineffective disaster response}

% Disaster disproportionately effect the poor and economically vulnerable
% Can amplify existing inequalities in communities that don't have survellance or data collection. Or help mitagte it due to easier data collection
% Talking about how it could include socioeconomic data into the disaster response information (paper example)
% (paper example) from all of the introductions and discussion sections.

\section{Application of remote sensing in disaster response}

% This is the results section where we can describe the different application and some basic stats to show how well it can or cannot work (like in chinese mountain range paper)
% How we can use it to detect and classify damaged buildings (~3 paper examples)
% How we can use it for fire detection (2 paper examples)
% How we can use it for more general disaster detection (1 paper example)

AI techniques excel at finding complex non linear patterns in data which is otherwise differentiable from noise to humans or other pattern recognition tooling. It is not a huge surprise that AI then is capable of excelling at finding complex patterns in a huge variety of data leading to massive improvements over the previous state of art in a variety of areas (such as classification of risky icebergs in shipping routes).

\textbf{Example 1- Wildfires}

Wildfire detection is an example of disaster detection where AI has been used since the 1990s. The authors of the paper begin with a review of ways various types of remote sensing data have led to improvements in rapid response systems. 

Over time, researchers have explored increasingly more complex data signal sources- from one dimensional inputs such as thermometers, to using increasingly higher resolution satellite cloud data, to complex sensing tools which have high dimensional outputs (which can be helpful where, for example, clouds might block the forests from view of the satellite imagery). This issue of cloud cover blocking data is quite a big problem for systems entirely reliant on computer vision techniques. 

\textbf{"Matching VIIRS forest fire location data with historical ground forest fire data shows that less than 30\% of forest fires were detected by satellite, and lightning strikes account for less than 15\% of forest fires."}
(Chinese mountain paper)

This shows a common theme in AI in general- that the quality of output system is dependent both on having high quality data, and also  designing systems to fully take advantage of the data that you do have. With this in mind, the paper sets out the goal of combining the strengths of computer vision techniques and transformer inspired mechanisms for more comprehensive analysis of wildfire patterns in a ConvNeXt-based architecture. System values extracting a lot of meaning from less data to build more scalable systems.

\textbf{Algorithm design}
"To reduce computational complexity while effectively capturing both spatial and channel-wise features, depthwise separable convolutions are utilized. Instead of Batch Normalization, Layer Normalization is applied, providing greater stability during training, especially with large datasets. The GELU activation function is employed to improve gradient flow and introduce better non-linearity, resulting in richer feature representations. Furthermore, the use of larger kernel sizes enables improved context aggregation, emulating the global receptive field characteristic of transformers."



\textbf{Example 2- Iceberg classification}

A more commercially focused project was using AI for detecting icebergs in Northern shipping routes. Traditional classifiers have been built on relying on principles which may not be true in all cases (such as icebergs that exhibit higher backscatter than the surrounding water). 

Due to the enormous range of weather, lighting, sea,  and the physical conditions of the icebergs themselves, it has proven to be an enormously complex task to create a robust classifier for this. Conversely, such a system would be enormously valuable to the shipping industry, and therefore they are strongly motivated to invest in improving the technology.

Another note about about this is the example is the complexity of getting good data to begin with. In the one year study, only 102 images in the required format were able to be collected which then all required extensive analysis by GIS specialists and then confirmed with rigorous cross referencing with other data sources. The paper uses this human labeled data to build a CNN model, which performs well compared to the limitations of previous techniques.




\section{Challenges of using remote sensing for disaster response}

\subsection{Data availability}

% Using super sampling to make low res data more useful (paper example)
% Using genAI to make up synthetic data (paper example)

\subsubsection{Compute constraints}

% How it can be hard to run large models on limited hardware and bandwidth
% Embedded processing due to bandwidth contraints from satellites (paper example)
% Compute constraints on the ground and speed requirements (paper example)

\section{Conclusion}

% Summarise the main points
% That we can use remote sensing for disaster response
% That it can be great but there are still challenges
% The current challenges that it face
% Touch on future directions (like end to end disaster response, smaller models, more data sources etc)

\bibliographystyle{IEEEtran}
\bibliography{references}
  
\end{document}
