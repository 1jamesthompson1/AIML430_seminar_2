\documentclass[conference,a4paper]{IEEEtran}
\IEEEoverridecommandlockouts
% The preceding line is only needed to identify funding in the first footnote. If that is unneeded, please comment it out.
%Template version as of 6/27/2024

\usepackage{cite}
\usepackage{amsmath,amssymb,amsfonts}
\usepackage{algorithmic}
\usepackage{graphicx}

\usepackage{textcomp}
\usepackage{tikz}
\usepackage{booktabs}
\usepackage{xcolor}
\usepackage{hyperref}
\usepackage{array}
\usepackage{tabularx}
\usepackage{amssymb} 


\def\BibTeX{{\rm B\kern-.05em{\sc i\kern-.025em b}\kern-.08em
    T\kern-.1667em\lower.7ex\hbox{E}\kern-.125emX}}
\begin{document}

% Custom figure command: pass filename and caption
\newcommand{\cfigure}[2]{%
  \begin{figure}[h]
    \centering
    \includegraphics[width=\linewidth]{figures/#1.png}%
    \caption{#2}%
    \label{fig:#1}%
  \end{figure}%

}
\title{AI for disaster detection and response using satellite imagery}

\author{Dwayne Mark Acosta (300665276) \\ Mohamed Amine Benaziza (300684553) \\ David Franz (300360491) \\ Ray Marange (300671115) \\ James Thompson (300680096)\\
\textit{Victoria University of Wellington}\\}
\date{\today}

\maketitle

\section{Introduction}

% Introduce this paper, It is a summary of ai disaster detection and problems using 5 primary papers (but hopefully more) ...
%% introduce the what remote sensing is
%% tell some statistics about how disaster are bad and increasing (the papers have lots of references)
%% more or less show that there is a great opportunity and need for better disaster response

\cite{elbohy2025fusion} etc...

\title{\bfseries\large Implications of Effective or Ineffective Disaster Response}
\author{Ray Marange}
\date{}

\begin{document}

\maketitle

\section{Implication of Effective or Ineffective Disaster Response}

Artificial Intelligence (AI) presents both promise and pitfalls in disaster response. Its effectiveness is shaped not only by technical precision but also by contextual relevance, equity, and data accessibility.

\subsection*{Key Studies and Their Implications}

\begin{table}[h!]
\centering
\begin{tabular}{|P{3.5cm}|P{5cm}|P{5cm}|}
\hline
\textbf{Study} & \textbf{Effective Response} & \textbf{Risks of Ineffective Use} \\
\hline
CV + Satellite Imagery \cite{paper2} & Enables rapid damage quantification and targeted aid deployment & Misclassification in informal settlements leads to inequitable aid distribution \\
\hline
Generative AI for YOLO \cite{paper3} & Augments data scarcity, improving detection accuracy & Synthetic data may fail to generalize across diverse real-world contexts \\
\hline
CNN + Transformer for Wildfires \cite{paper4} & Supports early detection and timely evacuation & Blind spots in remote zones delay alerts and response \\
\hline
ESRGAN for Damage Detection \cite{paper6} & Enhances image resolution, boosting model accuracy and trust & Poor image quality risks biased assessments and misinformed decisions \\
\hline
Lightning-Caused Fires in China \cite{paper1} & Regional pattern analysis improves model relevance and responsiveness & Generic models misalign with local environmental and social needs \\
\hline
\end{tabular}
\caption{Comparative analysis of AI applications in disaster response}
\label{tab:ai_disaster_response}
\end{table}

\subsection*{Synthesis of Implications}

Effective AI-driven disaster response systems are characterized by:

\begin{itemize}
    \item \textbf{Timeliness}: Early detection enables rapid mobilization and evacuation.
    \item \textbf{Accuracy}: Augmentation and hybrid models improve damage assessment.
    \item \textbf{Trustworthiness}: Explainable AI fosters confidence among stakeholders.
    \item \textbf{Context-Awareness}: Regional and socioeconomic data enhance relevance.
    \item \textbf{Equity-Driven Design}: Inclusive models ensure vulnerable populations are not excluded.
\end{itemize}

Conversely, ineffective deployment of AI can exacerbate vulnerabilities, delay aid, and reinforce systemic inequalities. The implications are profound—technological decisions directly shape human outcomes in crisis scenarios.

\bibliographystyle{IEEEtran}
\bibliography{references}
  
\end{document}
