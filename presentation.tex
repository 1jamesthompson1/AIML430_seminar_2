\documentclass{beamer}
\usepackage[utf8]{inputenc}
\usepackage{graphicx}
\usepackage[T1]{fontenc}
\usepackage{lmodern}
\usepackage{tabularx}
\usepackage{booktabs}
\usepackage[backend=biber, style=verbose]{biblatex}
\addbibresource{references.bib}

\usetheme{Madrid}
\usecolortheme{seahorse}
\usefonttheme{professionalfonts}
\usepackage{xcolor}
\definecolor{brand}{HTML}{1F77B4}
\definecolor{accent}{HTML}{E15759}
\setbeamercolor{title}{fg=brand}
\setbeamercolor{frametitle}{fg=brand}
\setbeamercolor{structure}{fg=brand}

\usepackage{fontawesome5}


\title{AI for disaster detection and response}
\author{Dwayne Mark Acosta (300665276) \\ Mohamed Amine Benaziza (300684553) \\ David Franz (300360491) \\ Ray Marange (300671115) \\ James Thompson (300680096)}
\date{\today}

\newcommand{\namedframe}[3]{
  \begin{frame}
    \frametitle{#2}
    \framesubtitle{#1}
    #3
  \end{frame}
}
\newcommand{\sectionframe}[2]{
  \begin{frame}
    \begin{center}
      \usebeamerfont{title}\usebeamercolor[fg]{title}\Huge #2\\
      \vspace{2em}
      \normalsize Presented by \\
      \vspace{0.5em}
      \usebeamerfont{frametitle}\usebeamercolor[fg]{frametitle}\LARGE
      #1
    \end{center}
  \end{frame}
}

\begin{document}

\frame{\titlepage}

\namedframe{James Thompson}{Introduction: Disaster reponse}{
  \begin{columns}[T,totalwidth=\textwidth]
    \begin{column}{0.5\textwidth}
      Natural disasters are increasing in frequency and severity. In 2023 alone more than 93 million people were affected and almost 100,000 people killed.\footnotemark

      \pause
      \vspace{1em}

      An effective disaster detection and \alert{response} is crucial to \textbf{mitigate} and reduce impact.

    \end{column}
    \pause
    \begin{column}{0.5\textwidth}
      
      \begin{block}{Ground-based assessments are often}
        \begin{itemize}[<+->]
          \item[\faHourglassHalf] \textbf{Slow} --- access and coverage take time
          \item[\faExclamationTriangle] \textbf{Dangerous} --- unsafe conditions for responders
          \item[\faSearchMinus] \textbf{Limited in scope} --- hard to get a wide-area view
        \end{itemize}
      \end{block}
    \end{column}
  \end{columns}
  \footnotetext{\fullcite{uclouvain/credEmergencyEventsDatabase2025}}
}

\namedframe{James Thompson}{Introduction: Remote sensing}{
  \begin{minipage}{\textwidth}
    \centering
    Remote sensing allows \textbf{data collection} from a \textbf{distance} (e.g., satellites, aircraft), providing a \alert{broad} and \alert{rapid} overview of disaster-affected areas.

    \pause
    \vspace{1em}
    \textbf{AI} can \textbf{automate} and \textbf{accelerate} the analysis of this data, extracting \alert{valuable insights} to inform disaster response efforts in short and long time frames.
  \end{minipage}

  \pause
  \vspace{2em}

  \begin{columns}[T,totalwidth=\textwidth]
    \begin{column}{0.48\textwidth}
      \textbf{Flood damaged classification}
      \small Can reliably achieve high accuracy\footnotemark in classifying flood-damaged buildings using satellite imagery and deep learning models
    \end{column}
    \pause
    \begin{column}{0.48\textwidth}
      \textbf{Fire detection}
    \small Does work but not consistently across different environments and methods.
    \end{column}
  \end{columns}
  \footnotetext{As in around 90\% accuracy in multiple independent studies and methods}
}

\namedframe{James Thompson}{Introduction: This presentation}{
  \begin{columns}[T,totalwidth=\textwidth]
    \begin{column}{0.32\textwidth}
      \textbf{Implications}\\
      \small Discuss the importance of effective disaster response and the potential impacts of using AI in this process.
    \end{column}
    \pause
    \begin{column}{0.32\textwidth}
      \textbf{Applications}\\
      \small Explore specific applications of AI and remote sensing in disaster response, such as classifying damaged buildings and detecting wildfires.
    \end{column}
    \pause
    \begin{column}{0.32\textwidth}
      \textbf{Challenges}\\
      \small Address the challenges of using AI for disaster response, focusing on data availability and compute constraints.
    \end{column}
  \end{columns}
}

\sectionframe{Ray Marange}{Implications}

\namedframe{Ray Marange}{Implications: The Dual Edge of AI in Disaster Response}{
\begin{table}[]
  \scriptsize
  \centering
  \begin{tabular}{p{2.2cm} p{4.5cm} p{4.5cm}}
    \toprule
    \textbf{Focus Area} & \textbf{If Effective} & \textbf{If Ineffective} \\
    \midrule
    Damage Assessment & Enables rapid, targeted aid delivery & Excludes vulnerable populations \\
    \midrule
    Data Scarcity & Solves data shortages; generalizes to new zones & Creates brittle, failing models \\
    \midrule
    Early Detection & Enables early alerts for timely evacuations & Causes dangerous delays due to blind spots \\
    \midrule
    Image Clarity \& Trust & Builds accuracy and trust with explainable tools & Leads to biased outcomes in low-resource regions \\
    \midrule
    Context Matters & Context-aware for accurate interventions & Generic models misdirect resources \\
    \bottomrule
  \end{tabular}
\end{table}
}

\namedframe{Ray Marange}{Implications: Principles for Effective AI Implementation}{

\begin{enumerate}
  \item \textbf{Timely} \\ Enable early detection and rapid mobilization

  \item \textbf{Accurate} \\ Leverage advanced techniques for precision

  \item \textbf{Trustworthy} \\ Incorporate explainability to build confidence

  \item \textbf{Context-Aware} \\ Integrate local socioeconomic and geographic data

  \item \textbf{Equity-Driven} \\ Ensure no one is left behind due to data gaps
\end{enumerate}
}

\namedframe{Ray Marange}{Implications: Key Takeaways}{
\textbf{The Bottom Line}
\begin{itemize}
  \item AI that aligns with these principles can mitigate disasters without amplifying inequalities
  \item AI that ignores them risks turning technology into a liability
  \item Effectiveness depends on balancing technical precision with social awareness
  \item Context-specific implementation is crucial for success
\end{itemize}
}

\sectionframe{David Franz}{Applications}

% Applications
\namedframe{David Franz}{Applications: Overview}{
    \begin{itemize}
        \item Artificial intelligence techniques excel at finding complex non linear patterns in complex data which is otherwise differentiable from noise to humans
        \item Traditional methods fail under certain circumstances, but modern CNN and transformer based deep learning have been able to efficiently learn patterns from complex combinations of sensor data
        \item We examine applications in wildfire detection, flood response and post response
    \end{itemize}
}

% Applications
\namedframe{David Franz}{Applications: Wildfires}{
    \begin{itemize}
        \item Wildfire detection is an example of disaster detection where AI has been used since the 1990s, but over time researchers have explored increasingly more complex input data.
        \item Computer vision techniques using satellite imagery has lead to impressive results in some cases, but cloud cover blocking data is a big problem, with systems entirely reliant on computer vision techniques only correctly detecting fires on historical satellite data in 30\% of cases.
    \end{itemize}
}

% Applications
\namedframe{David Franz}{Applications: Wildfires}{
  \textbf{ConvNeXt Approach}
    \begin{itemize}
        \item The paper sets out the goal of combining the strengths of computer vision techniques and transformer inspired mechanisms for more comprehensive analysis of wildfire patterns in a ConvNeXt-based architecture. 
        \item The model is carefully designed to trade of computational complexity and accuracy, and many different model variations were released for different hardware types and use cases (ranging from super computer to in field local models).
        \item         This is able to combine many different sources of sensor data and extract meaningful patterns and achieve state of the art performance compared to other recent models.
    \end{itemize}
}

% Applications
\namedframe{David Franz}{Applications: Flooding}{
    \begin{itemize}
        \item Flooding is the most frequent natural disaster causing widespread loss of life and destruction of property. 
        \item Some flooding will develop slowly, but the most dangerous kind (flash floods) can develop very quickly, leaving minimal time to evacuate or otherwise prepare. This is an area then where small improvements can translate to significant reduction of casualties. 
    \end{itemize}
}

% Applications
\namedframe{David Franz}{Applications: Flooding}{
\textbf{Human Impact and Economics}
    \begin{itemize}

        \item Certain geographic areas which experience wet seasons are particularly susceptible to this danger- the paper focuses primarily on Malaysia, (of which, flooding accounts for 60\% of it's total occurrences of natural disasters), but mentions that globally 1.81 billion people are exposed to flood risk. Beyond the human impact, flooding can have a significant effect on the economy of countries.
        \item " In 2021 alone, water-related disaster resulted in economic losses amounting to \$224.2 billion  globally, nearly doubling the annual average of \$117.8 billion recorded from 2001 to 2020. A recent report also predicts that floods alone are expected to erode more than \$2 trillion from the global GDP by 2050."

    \end{itemize}
}

% Applications
\namedframe{David Franz}{Applications: Flooding}{
  \textbf{YOLO Approach}
    \begin{itemize}
        \item The paper tries an approach of building a computer vision based classifier using a dataset of 3750 images collected from various generative AI tools, which were then preprocessed to get a final dataset of 6374 images of flood scenarios. 
        \item The prompts used to specify the images included specifying the expected country (Malaysia) and specified the angles they knew they would need for the classifier. 
        \item This is a uniquely low cost method of gathering a comprehensive dataset. In the context of Malaysia, rural areas are more highly effected by flash flooding, and in general flooding has a higher impact on areas with higher poverty. Such a low cost method of gathering data would be particularly valuable to areas with higher poverty.
    \end{itemize}
}

% Applications
\namedframe{David Franz}{Applications: Flooding}{
\textbf{YOLO Results}
    \begin{itemize}
        \item The final results are good, but not beating the state of the art with a mAP of 0.79 and F1-score of 90.8 (in comparison to Mask-R-CNN achieving mAP of 0.95 and F1-score of 97). 
        \item The uniquely low cost way of building the dataset combined with fairly strong performance do indicate that this is a promising direction to explore, and the paper suggests future research directions which they believe would improve performance further. 
        \item Particularly in areas of higher poverty, non synthetic datasets may not even exist, so understanding how to get good performance from synthetic data may help reduce inequities associated with this.
    \end{itemize}
}


% Applications
\namedframe{David Franz}{Applications: Flooding}{
\textbf{Post flood water damage}
    \begin{itemize}
        \item Another aspect of flood response is the post flooding stage. The YOLO based model is aims to improve detection of the flood itself, whereas the post flooding stage is based on fixing the problems caused by the flood.
        \item This paper builds a classifier using satellite imagery of buildings pre and post flood with simple labels to indicate different sorts of damage.  
        \item         The paper suggests particular architectural designs leading to their success such as separate encoders developed for pre and post disaster imagery, and also pre training on non disaster satellite imagery. 
        \item The model achieved strong performances of 91.4\% accuracy. 
    \end{itemize}
}

\sectionframe{Dwayne Acosta}{Challenges}

% Slide 1,2,3: Technical & Deployment Challenges
\namedframe{Dwayne Acosta}{Challenges: Technical \& Deployment}{
\begin{itemize}
    \item \textbf{1. Resource-Constrained Deployment and Real-Time Processing}
    \begin{itemize}
        \item Deep learning models require substantial computational resources, making deployment on UAVs and embedded systems challenging due to limited processing power, strict energy constraints, and restricted communication bandwidth (Elbohy et al., 2025)
        \item Example: ConvNeXt achieved 99.05\% accuracy for wildfire detection but lacks practical deployment strategies for edge devices and integration with emergency alert systems
        \item Critical trade-off: high-accuracy models often exceed hardware capabilities, forcing compromises between detection performance and operational feasibility
    \end{itemize}
\end{itemize}
}

\namedframe{Dwayne Acosta}{Challenges: Technical \& Deployment}{
\begin{itemize}
    \item \textbf{2. Model Interpretability and Trust Issues}
    \begin{itemize}
        \item DL models operate as "black boxes," providing no transparency into their decision-making process for critical damage assessments
        \item Trust issues: Emergency responders hesitate to act on AI predictions when lives are at stake without understanding the reasoning
        \item Explainability attempts fall short: Grad-CAM++ visualizations often highlight irrelevant regions rather than actual structural damage, failing to provide meaningful explanations (Lagap \& Ghaffarian, 2025)
    \end{itemize}
\end{itemize}
}


\namedframe{Dwayne Acosta}{Challenges: Technical \& Deployment}{
\begin{itemize}
    \item \textbf{3. Cross-Event Generalization and Class Imbalance}
    \begin{itemize}
      \item Kim et al. (2022) investigated the application of computer vision techniques for water-related disaster damage assessment using satellite imagery
      \item Performance drops: 85.9\% (in-domain) → 80.3\% (out-of-domain), indicating models struggle to generalize across different disaster contexts despite strong performance on familiar events
      \item F1 scores fall below 0.5 for floods and tsunamis (Sunda Strait: 0.164, Midwestern Flooding: 0.495) primarily because submerged buildings become invisible in satellite imagery, causing systematic misclassification
      \item Class imbalance: "New buildings" (180) vs "Not damaged" (625) samples creates strong prediction bias, while submerged structures are completely absent from the Philippines post-Typhoon Haiyan dataset.
    \end{itemize}
\end{itemize}
}


% Slide 4: Data Acquisition & Quality Challenges
\namedframe{Dwayne Acosta}{Challenges: Data Acquisition \& Quality}{
\begin{itemize}
    \item \textbf{1. Cost, Accessibility \& Temporal Constraints}
    \begin{itemize}
        \item Very high-resolution satellite data is often limited, delayed, or cost-prohibitive due to cloud cover and operational constraints
        \item Trade-off dilemma: Higher spatial resolution satellites have longer revisit cycles that cannot meet disaster response timelines, while satellites with more frequent coverage lack the spatial detail needed for accurate damage assessment
        \item Promising solution: ESRGAN techniques can enhance low-resolution imagery, achieving 4-5\% accuracy improvements for damage detection without expensive very high-resolution data (Lagap \& Ghaffarian, 2025)
    \end{itemize}
\end{itemize}
}



% Slide 5: Data Acquisition & Quality Challenges
\namedframe{Dwayne Acosta}{Challenges: Data Acquisition \& Quality}{
\begin{itemize}
    \item \textbf{2. Atmospheric \& Environmental Limitations}
    \begin{itemize}
        \item VIIRS satellite detected only 14.8\% of 298 forest fires in Heilongjiang (2013-2020) due to cloud cover and transit timing (Jiao et al., 2023)
        \item Floods coincide with heavy cloud cover that obstructs Landsat and MODIS optical sensors
        \item Critical paradox: Disasters create the very conditions (smoke, debris, atmospheric disturbances) that compromise sensing capabilities when most needed

    \end{itemize}
\end{itemize}
}


% Slide 5: Operational & Methodological Challenges
\namedframe{Dwayne Acosta}{Challenges: Data Acquisition \& Quality}{
\begin{itemize}
    \item \textbf{1. Multi-Temporal Analysis \& Recovery Discrimination}
    \begin{itemize}
        \item Must track structural changes across pre-disaster → event → post-disaster periods to distinguish damage vs recovery vs new construction
        \item Models consistently misclassify partially damaged and recovered structures as seen in the Philippines Typhoon Haiyan dataset. Models are also unable to separate disaster damage from seasonal/urban changes
        \item Vision Transformers can compare all three temporal frames but show inconsistent temporal reasoning across damage categories (Lagap et al., 2025)
    \end{itemize}
\end{itemize}    
}



% Slide 5: Operational & Methodological Challenges
\namedframe{Dwayne Acosta}{Challenges: operational \& methodological}{
\begin{itemize}

    \item \textbf{2. Synthetic Data \& Training Limitations}
    \begin{itemize}
        \item Generated imagery achieves high metrics (96.7\% mAP50, 90.8\% F1) when using YOLO-based models on the Malaysian flood detection but suffers from cartoon-like artifacts requiring systematic filtering (Teoh et al., 2024)
        \item Volume restrictions and prompt engineering challenges limit scalability of synthetic data generation
        \item Cannot replicate complex physical disaster dynamics, environmental interactions, or temporal progression from pre to post-disaster
    \end{itemize}
\end{itemize}
}

\sectionframe{Amine Benaziza}{Conclusion}

\namedframe{Amine Benaziza}{Conclusion: Essentials \& highlights}{
\footnotesize
\begin{itemize}
  \item Remote sensing is central for rapid situational awareness; satellites often \emph{lead} ground reports by 1–8h.
  \item Main limits: observability (clouds/smoke/revisit), domain shift, label scarcity/imbalance.
  \item Fires: Lightning 77.6\%; VIIRS $<$30\% all/$<$15\% lightning; monitorable share $\downarrow$ with lightning ($r=-0.888$, $p=0.003$).
  \item Buildings: Pseudo-siamese grid $\sim$91\% in / $\sim$80\% OOD; Providencia 97.5\%, $F_1=0.851$.
  \item Floods/Wildfire/Polar/ESRGAN: YOLOv8 (synthetic) $\mathrm{mAP}_{50}=0.967$, $\mathrm{mAP}_{[0.5,0.95]}=0.787$, $F_1=0.908$ (field brittleness); ConvNeXt-S $\sim$99.05\%; RCM SAR+ViT+ERA5 $\rightarrow$ 96.5\% 4-way, 98.9\% target/no-target, $\sim$1\% false alarms; ESRGAN lifts small/minority-class recall.
\end{itemize}
}

\namedframe{Amine Benaziza}{Conclusion: Implications}{
\footnotesize
\begin{itemize}
  \item \textbf{Effective:} earlier wildfire alerts, faster flood triage, safer polar routing $\Rightarrow$ shorter time-to-aid \& less waste.
  \item \textbf{Ineffective:} synthetic$\to$real brittleness \& blind spots (clouds/informal settlements) $\Rightarrow$ inequitable aid/misallocation.
  \item \textbf{Principles:} \emph{Timely • Accurate • Trustworthy/Transparent • Context-aware • Equity-driven}.
  \item \textbf{Tactics:} uncertainty thresholds + explainability (e.g., Grad-CAM++); local ecological/socioeconomic priors; standardized capture/annotation.
\end{itemize}
}

\namedframe{Amine Benaziza}{Conclusion: Challenges \& Next steps}{
\footnotesize
\textbf{Challenges:}
\begin{itemize}
  \item \textit{Technical \& deployment:} edge compute/power/bandwidth; calibrated uncertainty \& interpretability.
  \item \textit{Data acquisition \& quality:} cost/access/latency; clouds/smoke occlusion; scarce/imbalanced labels.
  \item \textit{Operational \& methodological:} cross-event/region generalization; multi-temporal reasoning; synthetic$\leftrightarrow$real gap.
\end{itemize}
\textbf{Next steps:}
\begin{itemize}
  \item Task$\to$action pipelines with auditable decisions \& latency budgets.
  \item Distill/quantize (ConvNeXt-S, YOLOv8n/s/m) for robust edge deployment.
  \item Combing remote sensing modalities (SAR, optical, weather).
  \item Label-efficient learning + ESRGAN to lift minority-class recall.
  \item Standardized capture/annotations; fair evals (OOD, occlusion stress, latency \& uncertainty).
\end{itemize}
}



\end{document}
