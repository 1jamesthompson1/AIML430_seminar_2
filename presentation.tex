\documentclass{beamer}
\usepackage[utf8]{inputenc}
\usepackage{graphicx}
\usepackage[T1]{fontenc}
\usepackage{lmodern}
\usepackage{tabularx}
\usepackage[backend=biber, style=verbose]{biblatex}
\addbibresource{references.bib}

\usetheme{Madrid}
\usecolortheme{seahorse}
\usefonttheme{professionalfonts}
\usepackage{xcolor}
\definecolor{brand}{HTML}{1F77B4}
\definecolor{accent}{HTML}{E15759}
\setbeamercolor{title}{fg=brand}
\setbeamercolor{frametitle}{fg=brand}
\setbeamercolor{structure}{fg=brand}

\usepackage{fontawesome5}


\title{AI for disaster detection and response}
\author{Dwayne Mark Acosta (300665276) \\ Mohamed Amine Benaziza (300684553) \\ David Franz (300360491) \\ Ray Marange (300671115) \\ James Thompson (300680096)}
\date{\today}

\newcommand{\namedframe}[3]{
  \begin{frame}
    \frametitle{#2}
    \framesubtitle{#1}
    #3
  \end{frame}
}

\begin{document}

\frame{\titlepage}

\namedframe{James Thompson}{Introduction: Disaster reponse}{
  \begin{columns}[T,totalwidth=\textwidth]
    \begin{column}{0.5\textwidth}
      Natural disasters are increasing in frequency and severity. In 2023 alone more than 93 million people were affected and almost 100,000 people killed.\footnotemark

      \pause
      \vspace{1em}

      An effective disaster detection and \alert{response} is crucial to \textbf{mitigate} and reduce impact.

    \end{column}
    \pause
    \begin{column}{0.5\textwidth}
      
      \begin{block}{Ground-based assessments are often}
        \begin{itemize}[<+->]
          \item[\faHourglassHalf] \textbf{Slow} --- access and coverage take time
          \item[\faExclamationTriangle] \textbf{Dangerous} --- unsafe conditions for responders
          \item[\faSearchMinus] \textbf{Limited in scope} --- hard to get a wide-area view
        \end{itemize}
      \end{block}
    \end{column}
  \end{columns}
  \footnotetext{\fullcite{uclouvain/credEmergencyEventsDatabase2025}}
}

\namedframe{James Thompson}{Introduction: Remote sensing}{
  \begin{minipage}{\textwidth}
    \centering
    Remote sensing allows \textbf{data collection} from a \textbf{distance} (e.g., satellites, aircraft), providing a \alert{broad} and \alert{rapid} overview of disaster-affected areas.

    \pause
    \vspace{1em}
    \textbf{AI} can \textbf{automate} and \textbf{accelerate} the analysis of this data, extracting \alert{valuable insights} to inform disaster response efforts in short and long time frames.
  \end{minipage}

  \pause
  \vspace{2em}

  \begin{columns}[T,totalwidth=\textwidth]
    \begin{column}{0.48\textwidth}
      \textbf{Flood damaged classification}
      \small Can reliably achieve high accuracy\footnotemark in classifying flood-damaged buildings using satellite imagery and deep learning models
    \end{column}
    \pause
    \begin{column}{0.48\textwidth}
      \textbf{Fire detection}
    \small Does work but not consistently across different environments and methods.
    \end{column}
  \end{columns}
  \footnotetext{As in around 90\% accuracy in multiple independent studies and methods}
}

\namedframe{James Thompson}{Introduction: This presentation}{
  \begin{columns}[T,totalwidth=\textwidth]
    \begin{column}{0.32\textwidth}
      \textbf{Implications}\\
      \small Discuss the importance of effective disaster response and the potential impacts of using AI in this process.
    \end{column}
    \pause
    \begin{column}{0.32\textwidth}
      \textbf{Applications}\\
      \small Explore specific applications of AI and remote sensing in disaster response, such as classifying damaged buildings and detecting wildfires.
    \end{column}
    \pause
    \begin{column}{0.32\textwidth}
      \textbf{Challenges}\\
      \small Address the challenges of using AI for disaster response, focusing on data availability and compute constraints.
    \end{column}
  \end{columns}
}

\namedframe{Person name}{Implications}{

}

\namedframe{Person name}{Applications}{

}

\namedframe{Person name}{Challenges}{

}

\namedframe{Person name}{Conclusion}{

}


\end{document}
