\documentclass{beamer}
\usepackage[utf8]{inputenc}
\usepackage{graphicx}
\usepackage{tabularx}
\usepackage[backend=biber, style=verbose]{biblatex}
\addbibresource{references.bib}


\title{AI for disaster detection and response using satellite imagery}
\author{Dwayne Mark Acosta (300665276) \\ Mohamed Amine Benaziza (300684553) \\ David Franz (300360491) \\ Ray Marange (300671115) \\ James Thompson (300680096)}
\date{\today}

\newcommand{\namedframe}[3]{
  \begin{frame}
    \frametitle{#1}
    \framesubtitle{#2}
    #3
  \end{frame}
}

\begin{document}

\frame{\titlepage}

\namedframe{Persons name}{Introduction}{
  Introducing the topic of AI for disaster detection and response using satellite imagery.
}

\namedframe{Person name}{Implications}{
\subsection{EXPLORING GENERATIVE AI FOR YOLO-BASED OBJECT DETECTION}
\subsubsection{Effective response}
Generative augmentation mitigates data scarcity, allowing real-time object detection models like YOLO to perform reliably in novel disaster zones. This ensures immediate situational awareness for responders.

\subsubsection{Ineffective response}
Over-reliance on synthetic data without validation could produce fragile models that fail in real-world deployments, leading to delayed or incorrect interventions.

\subsection{FUSION OF CNN AND TRANSFORMER ARCHITECTURES FOR PROACTIVE WILDFIRE DETECTION}
\subsubsection{Effective response}
By detecting wildfires proactively, hybrid ConvNeXt-Transformer models allow earlier evacuations, reducing casualties and environmental damage. Real-time monitoring ensures continuous readiness.

\subsubsection{Ineffective response}
Failure to integrate proactive detection results in delayed alerts, where fires spread uncontrollably before responders can mobilize.

\subsection{ENHANCING POST-DISASTER DAMAGE DETECTION WITH ESRGAN}
\subsubsection{Effective response}
Super-resolution and explainability tools (Grad-CAM++) improve both technical accuracy and human trust in AI systems. They support not only immediate response but also long-term recovery monitoring, ensuring aid and rebuilding are data-driven.

\subsubsection{Ineffective response}
Ignoring image quality and explainability risks biased assessments, undermining both the technical system and the confidence of decision-makers who rely on it.

\subsection{FOREST FIRE PATTERNS AND LIGHTNING-CAUSED DETECTION IN CHINA}
\subsubsection{Effective response}
Regional pattern analysis complements cutting-edge detection systems, ensuring models are trained with ecological and geographic awareness. This increases accuracy and cultural relevance in response strategies.

\subsubsection{Ineffective response}
Neglecting local data means systems remain generic, overlooking regional fire regimes and producing interventions that are ill-suited to the actual context.

\subsection{SYNTHESIS}
Together, these studies show that effective disaster response requires:
\begin{itemize}
    \item \textbf{Timeliness} (early and real-time detection)
    \item \textbf{Accuracy} (data augmentation, super-resolution, hybrid models)
    \item \textbf{Trustworthiness} (explainability and contextual awareness)
    \item \textbf{Lifecycle coverage} (from detection to recovery)
\end{itemize}
The implications are clear: AI systems that align with these principles can transform disaster management, but gaps in any one dimension risk turning technology into a liability instead of an asset
}

\namedframe{Person name}{Applications}{

}

\namedframe{Person name}{Challenges}{

}

\namedframe{Person name}{Conclusion}{

}


\end{document}
